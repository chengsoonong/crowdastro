%!tex root=./thesis.tex
\chapter*{Abstract}
\label{cha:abstract}
\addcontentsline{toc}{chapter}{Abstract}

In this thesis we present a new supervised learning approach to the astronomical
problem of radio cross-identification. We focus in particular on training our
methods using crowdsourced data from the Radio Galaxy Zoo, a citizen science
website that allows volunteers to cross-identify radio objects.

Cross-identification is the problem of matching of objects detected at one
wavelength with objects detected at another. This is a particularly difficult
problem for radio, as a large number of radio objects are active galactic
nuclei, supermassive black holes at the centre of galaxies emitting huge radio
jets. These jets can sprawl across the sky in complex ways as they interact with
their environment and in general have no clear relationship with their host
galaxy.

In this thesis we have cast the radio cross-identification problem into a
machine learning context, framing it as a object localisation problem that can
be solved using binary classification. Using this framework, we then classified
galaxies in the Chandra Deep Field South according to whether they contain an
active galactic nucleus. We used a combination of image features and
astronomical features to represent each galaxy. Image features were extracted
from images of the radio sky using a convolutional neural network.

We trained the classifier using label data from Radio Galaxy Zoo. We found that
a classifier trained on these non-expert labels performs similarly to a
classifier trained on expert labels, attaining balanced accuracies of $(87.17 \pm 0.90)\%$ and $(88.74 \pm 0.77)\%$ respectively.

We investigated multiple ways of handling noise and redundancy in the
crowdsourced labels, and applied the classification model from the Raykar et al.
(2010) paper \emph{Learning From Crowds} to the Radio Galaxy Zoo labels.
Comparing this to a simple classification model trained on the majority vote, we
found that the majority vote approach obtained the highest classification
accuracy.

Finally, we investigated applications of active learning to the radio
cross-identification problem, with a focus on the Radio Galaxy Zoo project. We
applied query-by-committee to the radio cross-identification problem, finding
that query-by-committee outperforms random selection (but not if the random
selection is class-balanced). We then suggested an experiment to investigate
applications of active learning to Radio Galaxy Zoo, and highlighted problems
with current active learning literature when applied to citizen science.
