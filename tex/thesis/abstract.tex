%!tex root=./thesis.tex
\chapter*{Abstract}
\label{cha:abstract}
% \addcontentsline{toc}{chapter}{Abstract}

In this thesis we present a na\"ive, supervised learning approach to the problem
in astronomy of \emph{radio cross-identification}, training using crowdsourced
labels from the Radio Galaxy Zoo.

Cross-identification is the matching of objects detected at one wavelength with
objects detected at another. This is a particularly difficult problem for radio,
as a large number of radio objects are \emph{active galactic nuclei},
supermassive black holes at the centre of galaxies emitting huge radio jets.
These jets can sprawl across the sky in complex ways as they interact with their
environment, and in general have no clear relationship with their host galaxy.

We cast the radio cross-identification problem as an object localisation problem
which can be solved using binary classification. We use a combination of image
features and astronomical features to represent potential host galaxies of an
active galactic nucleus. Image features are extracted from images of the radio
sky using a convolutional neural network. We then classify galaxies in the
Chandra Deep Field South according to whether they contain an active galactic
nucleus.

The classifier is trained using label data from Radio Galaxy Zoo, a citizen
science website that allows volunteers to cross-identify radio objects. We find
that a classifier trained on these non-expert labels performs similarly to a
classifier trained on expert labels.

We investigate multiple ways of handling noise and redundancy in the
crowdsourced labels, and apply the classification model from \citet{raykar10} to
the Radio Galaxy Zoo labels. This is then compared to a simple classification
model trained on the majority vote. We find that the majority vote approach
obtains the highest classification accuracy.

Finally, we investigate applications of active learning to the radio
cross-identification problem, with a focus on the Radio Galaxy Zoo project. We
apply query-by-committee to the radio cross-identification problem, finding that
query-by-committee outperforms random selection (but not if the random selection
is class-balanced). We then highlight problems with current active learning
literature when applied to citizen science, and suggest an experiment to
investigate these problems further.
