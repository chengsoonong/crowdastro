%!TEX root=thesis.tex

\chapter{Introduction}
\label{cha:intro}

Observations of the sky in various wavelengths have lead to remarkable discoveries, from the precession of Mercury validating the general theory of relativity \todo{find a better example}, to hand-recorded supernovae showing that the universe's expansion is accelerating. As our observational power increases with newer and better telescopes, we are reaching a point where astrophysicists cannot process all of the data themselves --- for example, the upcoming Evolutionary Map of the Universe, a radio survey covering 75\% of the sky, is expected to find 30 times more radio galaxies than we have ever known before, bringing the total to over 70 million. One hope is that we can train new machine learning algorithms to classify and predict astrophysical objects based on the huge amounts of data found in these surveys.

This thesis presents a naïve, supervised learning approach to the problem of active galactic nuclei cross-identification, which forms part of the processing pipeline for radio surveys such as the Evolutionary Map of the Universe and the Australia Telescope Large Area Survey. Label data are sourced from the Radio Galaxy Zoo, a ``citizen science'' project that attempts to crowdsource the cross-identification problem on the FIRST and ATLAS surveys.

\todo{Figure out where to put these acknowledgements. Each of these has an acknowledgement requirement, so there's actually a paragraph I have to include for each of them.}

This thesis makes use of data products from
\begin{itemize}
    \item ATLAS
    \item WISE
    \item Radio Galaxy Zoo
    \item SWIRE
    \item Breast cancer Wisconsin dataset
    \item UCI Machine Learning Repository
\end{itemize}