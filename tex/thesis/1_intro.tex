%!TEX root=thesis.tex

\chapter{Introduction}
\label{cha:intro}

Observations of the sky in different wavelengths have lead to remarkable
discoveries and enhanced our knowledge of the universe, but with the development
of ever more powerful telescopes and data collection methods, we have already
reached a point where there is simply too much astronomical data to process and
analyse by hand. Faint Images of the Radio Sky at Twenty Centimeters (FIRST),
which began collecting data as early as 1993, detected around 946 000 distant
radio objects; the AllWISE catalogue, released in 2013, contains over 747
million mid-infrared objects.

such as the upcoming Square Kilometre Array
(SKA), we have reached a point where astrophysicists cannot process all of their
astronomical observations by hand.

The scale of modern astronomical surveys is enormous. The Evolutionary Map of
the Universe (EMU) will make use of the Australian SKA Pathfinder (ASKAP) radio
telescope to image 75\% of the sky in radio wavelengths. EMU is expected to find
30 times more radio galaxies than we have ever known before, bringing the total
to over 70 million.

Analysis of this dataset by hand will be intractable, so one hope is that we can
train new machine learning algorithms to classify and predict astrophysical
objects based on the huge amounts of data found in surveys like EMU.

This thesis presents a na\"ive, supervised learning approach to the problem of
active galactic nuclei cross-identification, which forms part of the processing
pipeline for radio surveys such as EMU and the Australia Telescope Large Area
Survey (ATLAS). Label data are sourced from the Radio Galaxy Zoo, a ``citizen
science'' project that attempts to crowdsource the cross-identification problem
on the FIRST and ATLAS surveys.

\todo{Figure out where to put these acknowledgements. Each of these has an acknowledgement requirement, so there's actually a paragraph I have to include for each of them.}

This thesis makes use of data products from
\begin{itemize}
    \item ATLAS
    \item WISE
    \item Radio Galaxy Zoo
    \item SWIRE
    \item Breast cancer Wisconsin dataset --- ``The database was obtained from the University of Wisconsin Hospitals, Madison from Dr. William H. Wolberg.''
    \item UCI Machine Learning Repository
\end{itemize}


\section{Contributions of this project}

\begin{itemize}
  \item ML on RGZ
  \item Non-physical model for cross identification
  \item Deep learning features for radio
  \item Effect of features. \url{https://see.stanford.edu/materials/aimlcs229/ml-advice.pdf}
  \item Framing cross identification as object localisation then binary classification
  \item Implementation of Raykar and Yan
  \item Compare Raykar, Yan, MV (wins)
  \item Compare Logistic Regression with Random Forest
  \item Benefit active learning
\end{itemize}
