%!TEX root=thesis.tex

\chapter{Introduction}
\label{cha:intro}

Observations of the sky in different wavelengths have led to remarkable
discoveries and enhanced our knowledge of the universe, but with the development
of ever more powerful telescopes and data collection methods, we have already
reached a point where there is simply too much astronomical data to process and
analyse by hand. The scale of modern astronomical surveys is enormous. Faint
Images of the Radio Sky at Twenty Centimeters (FIRST) \citep{becker95}, which
began collecting data as early as 1993, detected around 946 000 distant radio
objects; the AllWISE catalogue \citep{cutri13}, released in 2013, contains over
747 million mid-infrared objects.

Even better telescopes are still being developed. It is a particularly exciting
time in radio astronomy: The Square Kilometre Array (SKA) is expected to be
built by 2024 \citep{ska}, and it will produce over 160 terabytes of data per
second. Due to the scale of the SKA, many pathfinder projects have been launched
to provide testbeds for new technologies to be used in the SKA and for new ways
to handle the data that the SKA will produce. One such project is the Australian
SKA Pathfinder (ASKAP), a radio telescope in Western Australia. ASKAP will soon
be used to conduct the Evolutionary Map of the Universe (EMU), a wide radio
survey.

EMU will be huge and sensitive, imaging 75\% of the sky at sensitivities 40
times higher than the current largest northern sky radio survey. It is expected
to find 30 times more radio galaxies than we have ever known before, bringing
the total to over 70 million. Manual expert analysis of the EMU data will
clearly be intractable.

\begin{figure}
  \centering
  \includegraphics[width=0.8\textwidth]{images/herculesA.jpg}
  \caption{Hercules A, a radio galaxy. We see jets emitted from the supermassive
    black hole near the centre of the image. The galaxy itself is not visible in
    radio wavelengths. \emph{Image: B. Saxton, W. Cotton and R. Perley
    (NRAO/AUI/NSF)}}
  \label{fig:radio-galaxy}
\end{figure}

Objects detected by EMU will need to be cross-identified with their counterparts
in surveys at other wavelengths. Unfortunately, radio objects can be arbitrarily
complex --- many radio objects are jets from supermassive black holes at the
centre of galaxies, and these jets warp as they interact with their environment.
An example of such an object is shown in Figure \ref{fig:radio-galaxy}. An
estimated 10\% of EMU radio objects will be too complicated for current
automated cross-identification algorithms \citep{banfield15, norris11}.

\citet{banfield15} created Radio Galaxy Zoo to attempt to address this
problem. Radio Galaxy Zoo is a citizen science project based on the highly
successful Galaxy Zoo \citep{lintott08, lintott11}. The Zooniverse
platform\footnote{\url{http://zooniverse.org/}} created by Galaxy Zoo provides a
way for non-experts to help researchers label data across a wide range of
scientific fields, and has resulted in well over 110 publications. In the case
of Radio Galaxy Zoo, volunteers are invited to cross-identify radio objects
imaged by the NRAO Very Large Array and the Australia Telescope with their
infrared counterparts imaged by the Wide-area Infrared Survey Explorer (WISE)
and Spitzer telescopes. The cross-identification interface is available online
for anyone to use\footnote{\url{http://radio.galaxyzoo.org/}}. To date, with the
help of thousands of volunteers, Radio Galaxy Zoo has managed to cross-identify
over 100 000 radio galaxies. This still does not compare in scale to EMU, but
the hope is that these cross-identification labels can be used to train
next-generation machine learning algorithms.

This thesis presents a na\"ive, supervised learning approach to the problem of
active galactic nuclei cross-identification, using label data sourced from the
Radio Galaxy Zoo.

\section{Contributions}
\label{sec:contributions}

  \begin{itemize}
    \item ML on RGZ
    \item Non-physical model for cross identification
    \item Deep learning features for radio
    \item Effect of features. \url{https://see.stanford.edu/materials/aimlcs229/ml-advice.pdf}
    \item Framing cross identification as object localisation then binary classification
    \item Implementation of Raykar and Yan
    \item Compare Raykar, Yan, MV (wins)
    \item Compare Logistic Regression with Random Forest
    \item Benefit active learning
  \end{itemize}

\section{Outline}
\label{sec:outline}
  
  Chapter \ref{cha:astro} introduces astronomical concepts such as radio active
  galactic nuclei. These concepts are important to understand both the purpose
  of this thesis, and to understand the problem we are trying to solve. We also
  introduce four astronomical surveys --- EMU, ATLAS, WISE, and SWIRE --- that
  produced the data we used in our experiments.

  Chapter \ref{cha:ml} introduces machine learning concepts such as
  classification and image feature extraction. These concepts are the building
  blocks for a machine-learned algorithm for automated radio
  cross-identification. We also perform some experiments to test how selected
  crowd learning algorithms behave in different contexts.

  Chapter \ref{cha:cross-identification} brings together Chapters 2 and 3 to
  develop a machine learning approach to the radio cross-identification task. We
  formalise the problem and highlight the problems that must be solved for
  development, trial different methods of approaching the task, and present
  results on task performance using our classifier.

  Chapter \ref{cha:active-learning} discusses active learning, and its
  application to crowdsourced projects like the Radio Galaxy Zoo. We perform
  some simple experiments and suggest future pathways for research in this area.

  Appendix \ref{cha:crowdastro} describes crowdastro, a Python package developed
  as part of this project. This package was used to perform all experiments
  described in this thesis. It also provides a command-line interface for
  training and executing the automated cross-identification task.

\section{Data Products and Packages Used in this Thesis}
\label{sec:data-products}
  
  The work presented in this thesis would not have been possible without the
  various data products used for experiments and development.

  This thesis makes use of data products from the Wide-field Infrared Survey
  Explorer (WISE), the Spitzer Space Telescope, and the Australia Telescope
  Compact Array. WISE is a joint project of the University of California, Los
  Angeles, and the Jet Propulsion Laboratory/California Institute of Technology,
  funded by the National Aeronautics and Space Administration. The Spitzer Space
  Telescope is operated by the Jet Propulsion Laboratory, California Institute
  of Technology, under contract with NASA. SWIRE was supported by NASA through
  the Spitzer Legacy Program under contract 1407 with the Jet Propulsion
  Laboratory. The Australia Telescope Compact Array is part of the Australia
  Telescope, which is funded by the Commonwealth of Australia for operation as a
  National Facility managed by CSIRO.

  This thesis also makes use of the Breast Cancer Wisconsin dataset. The dataset
  was obtained from the University of Wisconsin Hospitals, Madison from Dr.
  William H. Wolberg, and was accessed from the UCI Machine Learning Repository
  \footnote{https://archive.ics.uci.edu/ml/}.

  This thesis made use of astropy \citep{astropy}, a community-developed core
  Python package for astronomy, and scikit-learn \citep{scikit-learn}, an open
  source package machine learning package.

  Finally, this thesis had been made possible by the participation of more then
  10~000 volunteers in the Radio Galaxy Zoo project. Their contributions are
  individually acknowledged at http://rgzauthors.galaxyzoo.org.
