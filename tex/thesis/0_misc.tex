% Unsorted writing.

\chapter{Imagine There's No Chapters}
\label{cha:imagine}

\section{Papers Read}
\label{sec:papers}
  
  \begin{itemize}
    \item \citet{banfield15}
    \item \citet{yan10}
    \item \citet{yan11}
    \item \citet{dasgupta11}
    \item \citet{mozafari12} (re-reading in progress)
    \item \citet{fan15} (probably need to re-read)
    \item \citet{freund97} (need to re-read)
  \end{itemize}

\section{The Radio Galaxy Zoo}
\label{sec:rgz}

    Many galaxies contain a supermassive black hole in their centre. These
    black holes draw in surrounding matter, and may produce jets of matter as
    they do. These jets emit radio waves \todo{Or X-rays, I think, which are
    then redshifted to radio. Confirm and rewrite.} which can then be detected
    by radio telescopes. As black holes cannot be observed directly, this is
    the only way to identify black holes in distant galaxies. A radio-loud
    black hole in the centre of a galaxy is called an active galactic nucleus
    (AGN). \todo{I'm having some trouble finding information on AGNs,
    especially on their definitions. Find something concrete.} Large radio
    surveys such as Faint Images of the Radio Sky at Twenty-Centimeters (FIRST)
    \citep{white97, becker95} and the Australian Telescope Large Area Survey
    (ATLAS) \citep{franzen15} have found many sources of radio emissions, and
    these sources are dominated by AGNs \citep{banfield15}.

\section{Radio Galaxy Zoo Consensus Labels}
\label{sec:consensuses}
    
    Radio Galaxy Zoo volunteers are first asked to select combinations of radio
    objects that correspond to one radio source, and are then asked to select
    the location of the corresponding host galaxy \citep{banfield15}. Each radio
    subject is labelled by multiple volunteers. These labels are then collated
    as follows. First, the most common combination of radio objects is
    selected, and all labels that have a different combination are discarded.
    This radio combination is called the consensus radio combination. Then, the
    density of host location labels is estimated using Gaussian kernel density
    estimation (KDE), and the highest density location is selected. This is
    called the consensus host location. The consensus host location is then
    matched to the nearest infrared object.

    An alternative way to find the consensus host location is by using a
    clustering algorithm such as $k$-means. Host locations are clustered and
    the cluster containing the most locations is taken to represent the
    consensus; the consensus location is then the mean of the cluster. This is
    faster and more robust than using KDE, but requires $k$ to be known. $k$
    can be estimated using an algorithm such as PG-means \citep{hamerly07} or by
    choosing $k$ to minimise some information criterion \todo{cite:sklearn}.
    The consensus labels for the data associated with this thesis were found in
    this way, fitting a Gaussian mixture model to the host locations with the
    number of Gaussians chosen to minimise the Bayesian information criterion.

    Repeated volunteer labelling helps to reduce noise in the labels. This is
    necessary as the volunteers are not experts, and may incorrectly label the
    subject. The hope is that the majority of volunteers will correctly label
    subjects, which seems to be the case for radio subjects where more than
    75\% of volunteers agree \citep{banfield15}. The number of times a radio
    subject is shown to different volunteers is called the redundancy. The
    redundancy is 5 if the subject is a compact source, and 20 for all other
    sources. These numbers were chosen based on the redundancy levels of the
    original Galaxy Zoo project [Banfield, personal communication] \todo{Is
    this how to cite personal communication? Alternatively, is this written
    down somewhere?}. Since labels with radio combinations that disagree with
    the consensus are discarded, the redundancy is usually lower in practice
    when finding the host location. This can lead to very low redundancy input
    to KDE, causing KDE to fail. This failure can usually be caught, but the
    existing solution in this case is to take the mean of all host locations.
    This is not the consensus host location in general. Another effect is that
    since more complex sources have higher levels of disagreement in the radio
    combination stage, more complex sources have more discarded volunteer
    labels, and thus lower redundancy --- so more complex sources have more
    noise in their labels.

\section{Radio Cross-identification}
\label{sec:cross-identification}

    Each radio object has some associated infrared object called the host
    galaxy. The cross-identification task is to find the host galaxy given the
    radio object.
    
    For modelling the distribution, I have chosen to use logistic regression,
    i.e.
    \begin{equation}
        \label{eq:logistic-regression-cross-identification}
        p(y(x) = 1 \mid x, r) = \vec w \cdot \vec \phi(x, r)
    \end{equation}
    where $\vec \phi$ is a feature space mapping dependent on a galaxy and a
    radio object. The features should represent the galaxy in some way, so I
    have chosen the following feature space:
    \begin{equation}
        \label{eq:galaxy-features}
        \vec \phi(x, r) = \begin{pmatrix}
            \vec{\mbox{flux}}(x)\\
            \mbox{dist}(x, r)\\
            \vec{\mbox{cnn}}(\mbox{radio}(x))
        \end{pmatrix}
    \end{equation}
    $\vec{\mbox{flux}}(x)$ is a vector of infrared flux measurements of $x$,
    which can be obtained from the infrared survey catalogue. $\mbox{dist}(x,
    r)$ is the Euclidean distance across the sky between the centre of the $x$
    and the centre of $r$. $\vec{\mbox{cnn}}(m)$ is the output of the
    convolutional neural network on input image $m$, and $\mbox{radio}(x)$ is a
    $0.8' \times 0.8'$ image of the radio sky centred on $x$.

\section{Training Data}
\label{sec:training-data}
  
  The Crowdastro dataset is a set of training data for the binary
  classification problem described in Section \ref{sec:cross-identification}.
  The dataset contains features and labels for all objects detected in the WISE
  infrared survey. The prediction task is to predict the label of an object
  given its features.

  The features are not scaled and have not undergone any feature extraction
  process. They are the raw fluxes, distances, and radio images described in
  Section \ref{sec:cross-identification}.

  The labels are based on the consensus locations from the Radio Galaxy Zoo,
  matched to the nearest WISE object. WISE objects matched to a consensus
  location have the label $1$, and all other objects have the label $0$.
  Consensuses are found as described in Section \ref{sec:consensuses}, with
  consensus location decided by fitting a Gaussian mixture model. The number of
  Gaussians is found by a grid search minimising the Baysian information
  criterion.

