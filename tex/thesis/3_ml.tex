%!tex root=thesis.tex
\chapter{Machine Learning on Crowds}
\label{cha:ml}

\todo{Add a nice chapter summary.} In this chapter, I will introduce some
concepts from machine learning that I will use for the rest of this thesis. I
will discuss some common classification methods and techniques, image feature
extraction using convolutional neural networks, and methods to combat label noise.

\section{Classification}
\label{sec:classification}
    
    A common machine learning task, and indeed the machine learning focus of
    this thesis, is \emph{classification}. Given a set $\mathcal X$ of
    \emph{instances}, a set $\mathcal Y$ of \emph{labels}, the classification
    task is to assign a label $y \in \mathcal Y$ to each instance $x \in
    \mathcal X$. The ``true'' label of an instance is called the
    \emph{groundtruth}, and is denoted $z$. Classification thus amounts to
    modelling the conditional distribution $p(z \mid x)$. Some examples of
    classification include labelling images of digits, where $\mathcal X$ is a
    set of images and $\mathcal Y = \{0, \dots, 9\}$ \citep{lecun98}; and
    diagnosing breast cancer in patients, where $\mathcal X$ is a set of sets of
    medical observations and $\mathcal Y$ is the set $\{\text{malignant},
    \text{benign}\}$ \citep{wolberg90}.

    Modelling $p(z \mid x)$ requires some set of \emph{training data} $\mathcal
    D \subset \mathcal X \times \mathcal Y$. The training data are either used
    to find parameters for a model or to estimate the model directly. Both
    processes are called \emph{training} the model. The cardinality of the
    training data is denoted $N = |\mathcal D|$.

    \emph{Binary classification} is classification where each instance must be
    assigned one of two labels, i.e. $\mathcal Y \sim \{0, 1\}$. Instances with
    a true label of $0$ are called \emph{negative instances}, and instances with
    a true label of $1$ are called \emph{positive instances}. For this thesis, I
    will focus entirely on binary classification.

    Instances are usually represented by a vector of \emph{features}, denoted
    $\vec x$. Choosing features is an important and non-trivial part of
    modelling a classification task, and can strongly affect the ability of a
    model to classify instances. The dimensionality of the feature space is
    denoted $D$ and we assume that $\mathcal X = \mathbb{R}^D$.

    In this section, I will introduce two common ways of modelling binary
    classification: logistic regression, and random forests.

    \subsection{Logistic Regression}
    \label{sec:logistic-regression}

        Logistic regression is a simple linear model of binary classification.
        The conditional distribution $p(z \mid \vec x)$ is modelled as
        \[
            y(\vec x) = p(z = 1 \mid \vec x) = \sigma(\vec w^T \vec x + b),
        \]
        where $\vec w$ and $b$ are parameters to the model and $\sigma$ is the
        logistic sigmoid function
        \[
            \sigma(t) = \frac{1}{1 + \exp(-t)}.
        \]
        The elements of $\vec w$ are called the \emph{weights}, and control the
        effect of each feature on the output label probability. $b$ is the
        \emph{bias}, a constant added to all weighted sums of features
        independent of the features themselves. $b$ can be incorporated into
        $\vec w$ by adding an extra feature to $\vec x$ that is always $1$ and
        increasing the dimension of $\vec w$ by 1. I adopt this convention for
        the remainder of this thesis whenever $b$ is not explicitly included.

        Logistic regression is differentiable, and so may be trained using
        \emph{gradient descent}. In gradient descent, $\vec w$ is modified by
        the update equation
        \[
            \vec w \leftarrow \vec w - \lambda \nabla L(\vec w),
        \]
        where $L : \mathbb{R}^D \to \mathbb{R}$ is the \emph{loss} function and
        $\lambda \in (0, 1)$ is the \emph{learning rate}.

        The loss function represents how well the current model fits the
        observations. For binary classification, this is usually given by the
        \emph{cross-entropy error}
        \[
            L(\vec w) = -\frac{1}{N} \sum_{\vec x, z \in \mathcal D} \left(
                z \log y(\vec x) + (1 - z) \log (1 - y(\vec x))
            \right)
        \]
        with gradient
        \[
            \nabla L(\vec w) = -\frac{1}{N}
                \sum_{\vec x, z \in \mathcal D} \left(z - y(\vec x)\right).
        \]

    \subsection{Random Forests}
    \label{sec:random-forests}

\section{Feature Extraction from Images}
\label{sec:image-features}

\section{Crowdsourcing Labels}
\label{sec:crowdsourcing}

    For standard supervised learning tasks, the labels are generally
    provided by some \emph{expert}, carrying the assumption that these
    labels are correct and represent the groundtruth. More recently,
    however, many projects have \emph{crowdsourced} their labels: Non-expert
    people (the \emph{crowd}) volunteer or are paid to label data. The most
    obvious advantage of crowdsourcing is that crowds are able to cheaply or
    quickly label large amounts of data simply because there are so many
    people labelling. The crowd may be sourced from websites like Amazon
    Mechanical Turk\footnote{\url{http://mturk.com}} where small amounts of
    money are paid on a per-label basis, or they may volunteer out of
    interest in the labelling project.

    \subsection{Motivation}
    \label{sec:crowdsourcing-motivation}

        \todo{Write}

    \subsection{Examples}
    \label{sec:crowdsourcing-examples}

        Some notable examples of projects with crowdsourced labels include
        Galaxy Zoo \citep{lintott08}, a project to identify morphologies of
        galaxies from the Sloan Digital Sky Survey that has gathered tens of
        millions of labels for nearly 900 000 galaxies from over 80 000
        volunteers \citep{lintott11}; and Snapshot Serengeti \citep{swanson15},
        a project to observe mammals in Serengeti National Park that has
        classified nearly 11 million camera trap images with the help of 28 000
        volunteers.

        \todo{Rewrite and extend}

    \subsection{Problems}
    \label{sec:crowdsourcing-problems}

        Crowdsourcing is not without downsides. Since the crowd necessarily
        consists of non-experts, there is no longer any guarantee that labels
        are correct --- indeed, for the Radio Galaxy Zoo, balanced accuracy of
        individual labellers varies from 40--100\%, with an average of 71\%.
        Additionally, labels from different labellers may be correlated,
        different labellers may be better skilled at labelling different kinds
        of examples, some examples may be intrinsically hard for non-experts to
        label, and labellers may even be actively malicious in giving incorrect
        labels.

        \todo{Rewrite and extend}

\section{Aggregating Crowd Labels}
\label{sec:crowd-labels}

    \todo{Introduce}

    % There's kind of two things going on here - using multiple labels, and
    % using labels where the groundtruth doesn't exist. I should write about
    % that.

    In crowd learning scenarios, it is common to have multiple labels for a data
    point with no known groundtruth.

    \todo{extend}

    \subsection{Simulating a Crowd Labelling Task}

        \todo{Add preliminaries to make explanation of later experiments
        easier.}

    \subsection{Using Raw Labels}
    \label{sec:raw-labels}

        \todo{Write}

    \subsection{Majority Vote}
    \label{sec:majority-vote}

        One way to reduce label noise is to allow multiple labellers to label
        the same examples, and then for each example take the \emph{majority
        vote} to find a ``consensus'' label. If the label provided by the $t$th
        labeller is denoted $y_t$ and there are $T$ labellers, then the
        consensus label is given by
        \begin{equation*}
            y = \begin{cases}
                1 & \frac{1}{T} \sum_{t = 1}^T y_t > 0.5\\
                0 & \frac{1}{T} \sum_{t = 1}^T y_t < 0.5\\
            \end{cases}
        \end{equation*}
        with the remaining case decided evenly at random \citep{raykar10}.

        The idea is that the majority of labels are correct, so with
        sufficiently large amounts of relabelling we expect the label noise to
        be reduced. How large ``sufficiently large'' is is unclear and domain-
        dependent, and is outside the scope of this thesis, though papers like
        \citet{sheng08} and \citet{lin16} provide some information on how to
        choose relabelling rates. Of course, this fails in situations where the
        majority of labellers are not correct, which may be due to intrinsic
        difficulty or malicious labelling, or where there is no clear majority.

        \todo{Discuss weighted majority vote.}

    \subsection{Raykar et al. Model}
    \label{sec:raykar}

        One way to learn from crowdsourced labels when no groundtruth is
        available is to jointly model both the reliability of each labeller (the
        \emph{labeller model}) and the groundtruth itself (the
        \emph{classification model}).

        \citet{raykar10} proposed such a joint model, which we describe here.

        The accuracy of the $t$th labeller is modelled by by two parameters
        $\alpha_t$ and $\beta_t$. $\alpha_t$ is the \emph{sensitivity} and
        $\beta_t$ is the \emph{specificity} of labeller $t$, i.e.
        \begin{align*}
            \alpha_t &= p(y_t = 1 \mid z = 1)\\
            \beta_t &= p(y_t = 0 \mid z = 0).
        \end{align*}
        This model assumes that labeller reliability is independent of the
        example $x$, but dependent on the groundtruth label $z$. Note that under
        this model, the probability that labeller $t$ will assign a given label
        to a positive example is given by
        \begin{equation*}
            p(y_t \mid z = 1, \alpha_t) =
                (\alpha_t)^{y_t} (1 - \alpha_t)^{1 - y_t}
        \end{equation*}
        and the probability that they will assign a given label to a negative
        example is given by
        \begin{equation*}
            p(y_t \mid z = 0, \beta_t) =
                (\beta_t)^{1 - y_t} (1 - \beta_t)^{y_t}.
        \end{equation*}
        For ease of notation, let $\vec \alpha = (\alpha_1, \dots, \alpha_t)$
        and $\vec \beta = (\beta_1, \dots, \beta_t)$.

        With this method, the classification model can be any classifier.
        \citeauthor{raykar10} choose to use logistic regression:
        \begin{equation}
            \label{eq:raykar-logreg}
            p(z = 1 \mid \vec x, \vec w) = \sigma(\vec w^T \vec x).
        \end{equation}

        The model parameters $\vec \theta = \{\vec w, \vec \alpha, \vec \beta\}$
        can be found by maximising the likelihood. Under the assumption that
        examples are independently sampled and labellers are independent, the
        likelihood is given by
        \begin{align*}
            p(\mathcal D \mid \vec \theta) &=
                \prod_{i = 1}^N
                    \prod_{t = 1}^T
                        p(y_{t, i} \mid \vec x_i, \vec w, \alpha_t, \beta_t)\\
            &\begin{aligned}=
                \prod_{i = 1}^N
                    \prod_{t = 1}^T
                        &\bigg[p(y_{t, i} \mid z_i = 1, \alpha_t)
                            p(z_i = 1 \mid \vec x_i, \vec w)\\
                        &+ p(y_{t, i} \mid z_i = 0, \beta_t)
                            p(z_i = 0 \mid \vec x_i, \vec w)\bigg].
             \end{aligned}
        \end{align*}
        Since there are unknown values $z$, the maximum likelihood problem must
        be solved using expectation-maximisation. This has closed-form solutions
        for $\vec \alpha$ (Equation \ref{eq:raykar-alpha}) and $\vec \beta$
        (Equation \ref{eq:raykar-beta}), but gradient methods must be used for
        finding $\vec w$.

        $\mu_i$ is initialised with majority vote, i.e.
        \begin{equation*}
            \mu_i = \frac{1}{T} \sum_{t = 1}^T y_{t, i}.
        \end{equation*}
        The expectation step requires us to compute
        \begin{align*}
            a_i &= \prod_{t = 1}^T
                (\alpha_t)^{y_{t, i}} (1 - \alpha_t)^{1 - y_{t, i}}\\
            b_i &= \prod_{t = 1}^T
                (\beta_t)^{1 - y_{t, i}} (1 - \beta_t)^{y_{t, i}}\\
            \mu_i &\propto
                \frac{a_i \sigma(\vec w^T \vec x_i)}
                     {a_i \sigma(\vec w^T \vec x_i) +
                      b_i (1 - \sigma(\vec w^T \vec x_i))}.
        \end{align*}
        The maximisation step requires us to compute
        \begin{align}
            \label{eq:raykar-alpha}
            \alpha_t &= \frac{\sum_{i = 1}^N \mu_i y_{t, i}}
                             {\sum_{i = 1}^N \mu_i}\\
            \label{eq:raykar-beta}
            \beta_t &= \frac{\sum_{i = 1}^N (1 - \mu_i) (1 - y_{t, i})}
                            {\sum_{i = 1}^N (1 - \mu_i)}
        \end{align}
        and find the $\vec w$ that maximises the likelihood using gradient
        methods. In my implementation of this algorithm, I initialised $\vec w$
        using logistic regression trained on the majority vote, and then used
        the old $\vec w$ to initialise each subsequent maximisation step.

        As part of this thesis, I have produced an open source implementation of
        this algorithm, described in Section \ref{sec:crowdastro-raykar}. I then
        performed a series of experiments using this implementation, which I
        describe here.

        \subsubsection{Testing the Raykar Classifier}

            I first used the implementation to perform a simple, simulated crowd
            labelling problem. Five simulated labellers were assigned true positive
            and false positive rates uniformly distributed in the range $[0.25,
            0.75]$. Each simulated labeller labelled 50\% of the breast cancer
            dataset \citep{wolberg90} which can be obtained from the UCI Machine
            Learning Repository
            \citep{lichman13}\footnote{\url{https://archive.ics.uci.edu/m
            l/datasets/Breast+Cancer+Wisconsin+\%28Original\%29}}. Random samples of
            75\% of the examples were drawn 20 times and used to train both the
            \citeauthor{raykar10} classifier and a logistic regression classifier
            (using majority vote for the labels). These classifiers were tested
            against the groundtruth of the remaining 25\%. The results are plotted
            in Figure \ref{fig:raykar}. The Raykar classifier attained a balanced
            accuracy of $(58 \pm 8)\%$ and the logistic regression classifier
            attained a balanced accuracy of $(56 \pm 8)\%$.

            \begin{figure}[!ht]
                \centering
                \includegraphics[width=\textwidth]{images/experiments/raykar.pdf}
                \caption{Performance of the \citeauthor{raykar10} classifier against
                    logistic regression on a simulated crowd labelling of the breast
                    cancer dataset \citep{wolberg90}.}
                \label{fig:raykar}
            \end{figure}

        \subsubsection{The Effect of Class Imbalance}

            The second experiment tested the effect of class imbalance on the
            resulting balanced accuracy of the classifier, as well as on the
            predicted $\alpha$ and $\beta$ for each labeller. Scikit-learn's \citep
            {scikit-learn} \texttt{sklearn.datasets.make\_classification} function
            was used to generate datasets with 5-dimensional features (of which 3
            were informative, and 2 were redundant) with a class separation of 1 and
            1\% of true labels randomly flipped. Five datasets were generated, with
            different ratios of negative to positive examples. The ratios were
            $4:1$, $2:1$, $1:1$, $1:2$, and $1:4$. 1000 examples were drawn for each
            dataset. A crowd labelling task was then simulated, assigning 10
            labellers true positive and false positive rates uniformly at random in
            the interval $[0.5, 0.9]$ and allowing them to ``label'' the examples.
            The same labellers were used across all trials and datasets. The
            \citeauthor{raykar10} algorithm was then run on each dataset and the
            balanced accuracy and output $\alpha$ and $\beta$ values were recorded
            for each. The experiment was repeated five times. The resulting balanced
            accuracies are plotted against the ratios of negative to positive
            examples in Figure \ref{fig:raykar-class-balance-ba}, and the estimated
            $\alpha$ and $\beta$ values are plotted against the ratios in Figure
            \ref{fig:raykar-class-balance-ab}.

            \begin{figure}[!ht]
                \centering
                \includegraphics[width=\textwidth]
                    {images/experiments/raykar_class_balance_ba}
                \caption{Performance of the \citeauthor{raykar10} classifier on a
                    simulated labelling problem with different class imbalance.}
                \label{fig:raykar-class-balance-ba}
            \end{figure}

            \begin{figure}[!ht]
                \centering
                \includegraphics[width=\textwidth]
                    {images/experiments/raykar_class_balance}
                \caption{Output $\alpha$ and $\beta$ values of the
                    \citeauthor{raykar10} classifier on a simulated labelling
                    problem with different class imbalance.}
                \label{fig:raykar-class-balance-ab}
            \end{figure}

            The classifier performs well when many examples are positive, but
            performance is considerably reduced when there are more negative
            examples than positive examples. In these situations, the algorithm
            considerably underestimates $\alpha$, which may lead to ``mistrust'' of
            the crowd and hence reduce the balanced accuracy. The reason for the
            underestimation of $\alpha$ is unknown. Since we would expect the
            algorithm to perform identically under relabelling $0 \leftrightarrow
            1$, this may indicate a problem with the implementation.

        \subsubsection{The Effect of Label Noise}

            The final experiment was to determine the effect of label noise on the
            performance and output of the algorithm. 10 labellers were simulated
            with $\alpha$ ranging from $0.2$ to $1.0$ and $\beta$ fixed at $1.0$. A
            dataset was generated as for the previous experiment (but with balanced
            classes), the labellers were tasked with labelling this dataset, and the
            labels were used to train the algorithm. The balanced accuracy and
            $\alpha$ values were recorded. This experiment was repeated $5$ times,
            and repeated again with $\beta$ replacing $\alpha$. The results are
            shown in Figure

            \begin{figure}[!ht]
                \centering
                \includegraphics[width=\textwidth]{images/experiments/raykar_noise}
                \caption{Output $\alpha$, $\beta$, and balanced accuracy of the
                    \citeauthor{raykar10} classifier on a simulated labelling
                    problem with different amounts of label noise. ``Estimated
                    $\alpha$'' refers to the value of $\alpha$ output by the
                    algorithm, and $\alpha$ refers to the groundtruth. Filled areas
                    represent the standard deviation of multiple measurements.}
                \label{fig:raykar-noise}
            \end{figure}

            As expected, the balanced accuracy increases as label noise decreases.
            The output $\alpha$ and $\beta$ closely match the true values.

    \subsection{Yan et al. Model}
    \label{sec:yan}

        \todo{Clear this up with respect to the above examples.}

        \citet{yan10} propose modelling the $t$th labeller by a Bernoulli distribution of a data-dependent function $\eta_t(\vec x)$, parametrised as a logistic regression function $\eta_t(\vec x) = \sigma(\vec \omega_t^T \vec x + \gamma_t)$. The labeller model is thus
        \begin{equation*}
            p(y_t \mid \vec x, z, \vec \omega_t, \gamma_t) = \eta_t(\vec x)^{1 - |y_t - z|} (1 - \eta_t(\vec x))^{|y_t - z|}.
        \end{equation*}
        For ease of notation, let $\Omega = (\vec \omega_1^T, \dots, \vec \omega_T^T)$ and let $\vec \gamma = (\gamma_1, \dots, \gamma_T)$. This model can be obtained from the \citeauthor{raykar10} model by requiring $\alpha_t = \beta_t = \eta_t(\vec x)$ and allowing these parameters be data-dependent.

        As with \citeauthor{raykar10}, the classification model can be any classifier and \citeauthor{yan10} choose to use logistic regression (Equation \ref{eq:raykar-logreg}). The parameters $\vec \theta = \{\Omega, \vec \gamma, \vec w\}$ can be found by maximising the likelihood with expectation-maximisation. Under the assumptions that examples are independently sampled and that the labellers are independent, the likelihood is given by
        \begin{align*}
            p(\mathcal D \mid \vec \theta) &= \prod_{i = 1}^N \prod_{t = 1}^T p(y_{t, i} \mid \vec x_i, \vec w, \Omega, \vec \gamma)\\ &= \prod_{i = 1}^N \prod_{t = 1}^T \sum_{z = 0}^1 p(y_t \mid \vec x_i, z, \vec \omega_t, \gamma_t) p(z \mid \vec x_i, \vec w).
        \end{align*}
        The expectation steps requires us to compute
        \[
            \mu_i \propto \prod_{t = 1}^{T} p(y_{t, i} \mid \vec x_i, z_i = 1, \vec \omega_t, \gamma_t) p(z_i = 1 \mid \vec x_i, \vec w).
        \]
        The maximisation step requires us to maximise
        \begin{equation*}
            \sum_{i = 1}^N \sum_{t = 1}^T \sum_{z_i = 0}^1 p(z_i \mid \vec x_i, \vec w) (\log p(y_{t, i} \mid \vec x_i, z, \vec \omega_t, \gamma_t) + \log p(z \mid \vec x_i, \vec w))
        \end{equation*}
        with respect to $\vec w, \Omega,$ and $\vec \gamma$, where $p(z_i \mid \vec x_i, \vec w)$ is fixed to use the previous value of $\vec w$.