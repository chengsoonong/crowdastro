%!TEX root=thesis.tex
\chapter{The Radio and Infrared Sky}
\label{cha:astro}

    Modern astronomy relies on observations of deep space. Telescopes image the sky in different wavelengths, with different wavelengths carrying different physical meanings. In this chapter, I will talk about distant galaxies, supermassive black holes, and how they are detected and studied. I will discuss radio and infrared surveys, including their motivations, measurements, and physical interpretations. Finally, I will introduce two infrared surveys (SWIRE and WISE) and two radio surveys (FIRST and ATLAS).

    \section{Galaxies}



    \section{Supermassive Black Holes and Active Galactic Nuclei}

        Black holes are compact, dense objects of extraordinarily large mass. Due to their density, black holes have extremely strong gravitational fields, and any object that gets close enough will be unable to escape \citep{wald10}. This even includes light, which is effectively absorbed by black holes, giving them their name.

        \todo{Talk about how black holes are predicted by GR.}

            More information on black holes and general relativity can be found in \citet{wald10}.

        \todo{Talk about how we can observe black holes?}

        \subsection{Active Galactic Nuclei}

            Many galaxies contain a supermassive black hole in their centre. These black holes accrete mass from the surrounding galaxy, and in the process convert large amounts of matter into energy --- in some cases, up to $42\%$ of the matter is converted directly into energy \citep{alexander12}. This energy is emitted in the form of radio waves.



        These black holes draw in surrounding matter, and may produce jets of matter as they do. These jets emit radio waves \todo{Directly? Or is something else redshifted? Confirm and rewrite.} which can then be detected by radio telescopes. As black holes cannot be observed directly, this is the only way to identify black holes in distant galaxies. A radio-loud black hole in the centre of a galaxy is called an active galactic nucleus (AGN). \todo{I'm having some trouble finding information on AGNs, especially on their definitions. Find something concrete.} Large radio surveys such as Faint Images of the Radio Sky at Twenty-Centimeters (FIRST) \citep{white97, becker95} and the Australian Telescope Large Area Survey (ATLAS) \citep{franzen15} have found many sources of radio emissions, and these sources are dominated by AGNs \citep{banfield15}.

    \section{Infrared Surveys}

        \subsection{WISE}

        \subsection{SWIRE}

    \section{Radio Surveys}

        \subsection{FIRST}

        \subsection{ATLAS}

    \begin{enumerate}
        \item something about ATLAS, FIRST, EMU, Meerkat--MIGHTEE, WODAN
        \item something about SWIRE, WISE
    \end{enumerate}
