%!TeX root=thesis.tex

\appendix
\chapter{Crowdastro Package}
\label{cha:crowdastro}

As part of this thesis, we developed an open source Python package called
\emph{crowdastro}, containing methods for machine learning on the cross-
identification task, and implementations of many of the methods described here.
In this appendix, we briefly describe how to obtain this package, and list the
submodules available.

\section{Obtaining Crowdastro}

    The source for crowdastro is available on GitHub at
    \url{http://github.com/chengsoonong/crowdastro}. Crowdastro can also be
    installed through pip, by running \texttt{pip3 install crowdastro}. The code
    is MIT licensed.

\section{Crowdastro}

    The crowdastro package can be imported into Python or used with the command-
    line interface.

    \todo{Finish this.}

\section{Submodules}
    \label{sec:crowdastro-submodules}

    \subsection{crowdastro.active\_learning.random\_sampler}
        \label{sec:crowdastro-random-sampler}
    \subsection{crowdastro.active\_learning.sampler}
        \label{sec:crowdastro-sampler}
    \subsection{crowdastro.active\_learning.qbc\_sampler}
        \label{sec:crowdastro-qbc-sampler}
    \subsection{crowdastro.active\_learning.uncertainty\_sampler}
        \label{sec:crowdastro-uncertainty-sampler}

    \subsection{crowdastro.crowd.raykar}
        \label{sec:crowdastro-raykar}

        \texttt{crowdastro.crowd.raykar} is an implementation of the crowd
        learning algorithm developed by \citet{raykar10}, described here in
        Section \ref{sec:raykar}. The module provides a
        \texttt{RaykarClassifier} object which implements a modified scikit-
        learn interface.

    \subsection{crowdastro.crowd.util}
        \label{sec:crowdastro-util}

        \texttt{crowdastro.crowd.util} contains useful functions for dealing
        with crowd labels and performing related experiments shown in this
        thesis. These functions are:
        \begin{itemize}
            \item \texttt{balanced\_accuracy}, which computes the balanced
                accuracy of a classifier against a test set,
            \item \texttt{crowd\_label}, which simulates the crowd labelling
                task as described in Section \ref{sec:crowd-simulation},
            \item \texttt{majority\_vote}, which computes the majority vote of
                a set of crowd labels,
            \item \texttt{logistic\_regression}, a simple implementation of the
                logistic regression function (Equation
                \ref{eq:logistic-regression}).
        \end{itemize}

    \subsection{crowdastro.crowd.yan}
        \label{sec:crowdastro-yan}

        \texttt{crowdastro.crowd.yan} is an implementation of the crowd
        learning algorithm developed by \citet{yan10}, described here in
        Section \ref{sec:yan}. The module provides a
        \texttt{YanClassifier} object which implements a modified scikit-
        learn interface.
