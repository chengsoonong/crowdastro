%!TeX root=thesis.tex

\appendix
\chapter{Crowdastro Package}
\label{cha:crowdastro}

As part of this thesis, we developed an open source Python package called
\emph{crowdastro}, containing methods for machine learning on the
cross-identification task, and implementations of many of the methods described
here. In this appendix, we briefly describe how to obtain this package, and list
the submodules available.

\section{Obtaining and Using Crowdastro}

    The source code for crowdastro is available on GitHub at
    \url{http://github.com/chengsoonong/crowdastro}. Crowdastro can also be
    installed through pip, by running \texttt{pip3 install crowdastro}. The code
    is MIT licensed.

    The crowdastro package can be imported into Python or used with the
    command-line interface. Documentation for the command-line interface is
    available on Read the Docs at \url{https://crowdastro.readthedocs.io}.

\section{Submodules}
    \label{sec:crowdastro-submodules}

    In this section, we document the main submodules in crowdastro.

    \subsection{crowdastro.active\_learning}
    \label{sec:crowdastro-active-learning}

        \texttt{crowdastro.active\_learning} contains classes that simulate
        active learning tasks with different sampling methods. These all
        implement the same methods, and so can be used as drop-in replacements
        for each other.

        Samplers have access to a pool of unlabelled data and an array of known
        labels. They also have a \texttt{sample\_index} method that returns the
        index of an unlabelled data point to query an expert for the label, and
        a \texttt{sample\_indices} method that returns a list of such indices
        for bulk training. They also have an \texttt{add\_label} method to add
        the retrieved label to the array of known labels, and a
        \texttt{add\_labels} method which takes a list of such labels for bulk
        training.

        \subsubsection{qbc\_sampler}
        \label{sec:crowdastro-qbc-sampler}

            \texttt{crowdastro.active\_learning.qbc\_sampler} contains a class
            \texttt{QBCSampler} that simulates query-by-committee as described
            in Section \ref{sec:qbc}. The committee is composed of a
            user-defined number of logistic regression classifiers trained on a
            user-defined percentage of the labelled data. The
            \texttt{QBCSampler} also keeps a single logistic regression called
            the \emph{reference classifier} which is trained on all known
            labels; it is this classifier that is used to compute the balanced
            accuracy to avoid underreporting of accuracy due to sampling.

        \subsubsection{random\_sampler}
        \label{sec:crowdastro-random-sampler}

            \texttt{crowdastro.active\_learning.random\_sampler} contains
            classes that simulate an active learning task with passive sampling.
            There are two such classes:
            \begin{itemize}
                \item \texttt{RandomSampler}, which samples completely at
                    random,
                \item \texttt{BalancedSampler}, which samples evenly from binary
                    classes.
            \end{itemize}

        \subsubsection{sampler}
        \label{sec:crowdastro-sampler}

            \texttt{crowdastro.active\_learning.random\_sampler} contains the
            \texttt{Sampler} base class for other samplers.

        \subsubsection{uncertainty\_sampler}
        \label{sec:crowdastro-uncertainty-sampler}

            \texttt{crowdastro.active\_learning.uncertainty\_sampler} contains a
            class \texttt{ConfidenceUncertaintySampler} that simulates binary
            uncertainty sampling as described in Section
            \ref{sec:uncertainty-sampling}.

    \subsection{crowdastro.crowd}
    \label{sec:crowdastro-crowd}

        \texttt{crowdastro.crowd} contains classes for crowd learning. Both
        classes implement the same methods, and so can be used as drop-in
        replacements for each other. The module also contains some helper
        functions for crowd labels and related experiments.

        \subsection{raykar}
            \label{sec:crowdastro-raykar}

            \texttt{crowdastro.crowd.raykar} is an implementation of the crowd
            learning algorithm developed by \citet{raykar10}, described here in
            Section \ref{sec:raykar}. The module provides a
            \texttt{RaykarClassifier} object which implements a modified
            scikit-learn interface.

        \subsection{util}
            \label{sec:crowdastro-util}

            \texttt{crowdastro.crowd.util} contains useful functions for dealing
            with crowd labels and performing related experiments shown in this
            thesis. These functions are:
            \begin{itemize}
                \item \texttt{balanced\_accuracy}, which computes the balanced
                    accuracy of a classifier against a test set,
                \item \texttt{crowd\_label}, which simulates the crowd labelling
                    task as described in Section \ref{sec:crowd-simulation},
                \item \texttt{majority\_vote}, which computes the majority vote
                    of a set of crowd labels,
                \item \texttt{logistic\_regression}, a simple implementation of
                    the logistic regression function (Equation
                    \ref{eq:logistic-regression}).
            \end{itemize}

        \subsection{yan}
            \label{sec:crowdastro-yan}

            \texttt{crowdastro.crowd.yan} is an implementation of the crowd
            learning algorithm developed by \citet{yan10}, described here in
            Section \ref{sec:yan}. The module provides a \texttt{YanClassifier}
            object which implements a modified scikit-learn interface.

    \subsection{crowdastro.experiment}
    \label{sec:crowdastro-experiment}

        \texttt{crowdastro.experiment} contains all the experiments run for this
        thesis. The scripts in this module can be executed from the command
        line, e.g. \texttt{python3 -m crowdastro.experiment.experiment\_name}.