%!tex root=thesis.tex

\chapter{Conclusion}
\label{cha:conclusion}

    Ever larger and ever more detailed radio surveys will bring new challenges
    to astronomy. In this thesis we have focused on radio cross-identification,
    a task for which existing algorithms are expected to fail and for which a
    manual approach is intractable. We have presented a new, astronomical
    model-free, supervised learning approach to this problem. We hope that this
    approach will help guide the search for innovative ways to handle the data
    produced by the Evolutionary Map of the Universe.

    In Chapter \ref{cha:cross-identification}, we framed the
    cross-identification problem first as object localisation and then as binary
    classification of galaxies. We represented galaxies with the use of a
    convolutional neural network to extract features from ATLAS radio images,
    and combined these with features from the WISE telescope. We then tested a
    variety of methods of learning for a classifier on these features, including
    logistic regression, random forests, and the \citet{raykar10} crowd learning
    algorithm. We showed that logistic regression with majority vote
    outperformed the \citeauthor{raykar10} algorithm, and that a classifier
    trained on non-expert crowd labels attains comparable accuracy to a
    classifier trained on expert labels.

    We then looked at active learning in Chapter \ref{cha:active-learning},
    applying query-by-committee to the cross-identification task. We suggested
    an experiment for using active learning on Radio Galaxy Zoo, and highlighted
    some problems with existing active learning literature when applied to
    citizen science.

    Our work here is just the start of applying machine learning to radio
    cross-identification. In Section
    \ref{sec:cross-identification-conclusion-future-work} we have suggested a
    number of avenues for future work. Performance could be improved by
    incorporating hand-selected features, using information from more
    wavelengths, and developing a better feature extraction method. Our method
    could be extended to account for infrared-faint radio objects. Machine
    learning could be applied to learn to associate related radio objects with
    each other, and then with their host galaxy. Radio Galaxy Zoo volunteers
    could be analysed to determine the best possible labeller model for use in
    crowd learning algorithms. In Section \ref{sec:al-rgz-ideal-experiment} we
    suggested an experiment for applying active learning to Radio Galaxy Zoo and
    determining an appropriate uncertainty aggregation method for the radio
    cross-identification task. Finally, in Section \ref{sec:al-citizen-science},
    we highlighted some problems with active learning and crowd learning when
    applied to citizen science, suggesting that citizen science is a unique form
    of crowdsourcing that requires additional considerations.
