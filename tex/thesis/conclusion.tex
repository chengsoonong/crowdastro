%!tex root=thesis.tex

\chapter{Conclusion}
\label{cha:conclusion}

    In this thesis we have presented a supervised learning approach to the
    problem of radio cross-identification.

    In Chapter \ref{cha:cross-identification}, we framed the
    cross-identification problem first as object localisation and then as binary
    classification of galaxies. We represented galaxies with the use of a
    convolutional neural network to extract features from ATLAS radio images,
    and combined these with features from the WISE telescope. We then tested a
    variety of methods of learning for a classifier on these features, including
    logistic regression, random forests, and the \citet{raykar10} crowd learning
    algorithm. We showed that logistic regression with majority vote
    outperformed the \citeauthor{raykar10} algorithm, and that a classifier
    trained on non-expert crowd labels attains comparable accuracy to a
    classifier trained on expert labels.

    We then looked at active learning in Chapter \ref{cha:active-learning},
    applying query-by-committee to the cross-identification task. We suggested
    an experiment for using active learning on Radio Galaxy Zoo, and highlighted
    some problems with existing active learning literature when applied to
    citizen science.

    \todo{Finish this!}